\section{问题分析}
\subsection{freefall信号特性}
\par{1)对于单片机自身的加速度,主要是通过OLED板上的XYZ的输出来表示。借助课堂上所提供的工具FreeMASTER所能够反映出来的在自由落体的过程当中,XYZ数值的具体变化。因此,在针对单片机所进行的不同状态的运动时,可以通过观察所生成的图表,进而更直接的获得相应的特征数值变化。}

\begin{figure}[H]
\centering
\includegraphics[width=12cm]{figure/6.1.1.png}
\caption{FreeMASTER示波器界面} \label{fig:6.1.1}
\end{figure}

\par{2)针对freefall状态,我们小组先进行了多次的自由落体的测试,并且观察了三个数值所发生的变化。尽管每次下落时刻附近的波形在大体上相差很大,但是经过我们小组的分析,在波形突然开始随机变化前,有着一个状态,Z的数值猛然下降到与X的数值接近,而XY的数值基本上时没有变化的。}

\begin{figure}[htbp]
\centering

\subfigure{
\begin{minipage}[t]{0.5\linewidth}
\centering
\includegraphics[height=5cm]{figure/6.1.2.1.png}
\end{minipage}%
}%
\subfigure{
\begin{minipage}[t]{0.5\linewidth}
\centering
\includegraphics[height=5cm]{figure/6.1.2.2.png}
\end{minipage}%
}%

\centering
\caption{两次自由落体下波形图的状况(红为X,绿为Y,蓝为Z)}\label{fig:6.1.2}
\end{figure}

\par{3)理论上分析产生此种信号的原因,是因为单片机内部传感器是对其内部电子的相对加速度进行反馈。当静止不动时,电子保持1g的加速度,使得Z处在一个相对稳定的位置。当单片机自由落体时,单片机也是以1g的加速度下降,使得其相对加速度减小,趋近于0,所以使得反馈出来的数值接近在静止时XY的显示数值。}

\par{4)将范围扩展,测试的过程中主要是正着下坠所测得的参数,在实际状态的自由下落中,会出现不确定的下坠方式。所以,在自由下坠时,更为普适性的信号特性应该是其加速度的矢量和非常大的程度减小,反映到FreeMASTER则是在下落结束前一定会有一个参数剧烈下降。
}

\begin{figure}[htbp]
\centering

\subfigure{
\begin{minipage}[t]{0.4\linewidth}
\centering
\includegraphics[height=4cm]{figure/6.1.4.1.png}
\end{minipage}%
}%
\subfigure{
\begin{minipage}[t]{0.4\linewidth}
\centering
\includegraphics[height=4cm]{figure/6.1.4.2.png}
\end{minipage}%
}%

\centering
\caption{以其他方向自由落体的结果(红为X,绿为Y,蓝为Z)}\label{fig:6.1.4}
\end{figure}

\subsection{其他状况下的信号特性对比}
\par{1)冲击}
\par{在不同角度的冲击下,可以观察到,XYZ中总会出现至少一个数值大幅度的偏离中心值,即出现了该方向的一个很大的加速度,在总体上看,则是总加速度矢量的骤然增大。
}

\begin{figure}[htbp]
\centering

\subfigure{
\begin{minipage}[t]{0.33\linewidth}
\centering
\includegraphics[height=4cm]{figure/6.2.1.1.png}
\end{minipage}%
}%
\subfigure{
\begin{minipage}[t]{0.33\linewidth}
\centering
\includegraphics[height=4cm]{figure/6.2.1.2.png}
\end{minipage}%
}%
\subfigure{
\begin{minipage}[t]{0.33\linewidth}
\centering
\includegraphics[height=4cm]{figure/6.2.1.3.png}
\end{minipage}%
}%

\centering
\caption{在三个不同方向冲击下的结果(红为X,绿为Y,蓝为Z)}\label{fig:6.2.1}
\end{figure}

\par{2)静止}
\par{静止状态主要的特征是保持一个相对恒定的XYZ的输出值,在对于不同放置下的XYZ会有不同的对应的稳定状态值,总体上是三条保持平行的水平线}

\begin{figure}[htbp]
\centering

\subfigure{
\begin{minipage}[t]{0.33\linewidth}
\centering
\includegraphics[height=3cm]{figure/6.2.2.1.png}
\end{minipage}%
}%
\subfigure{
\begin{minipage}[t]{0.33\linewidth}
\centering
\includegraphics[height=3cm]{figure/6.2.2.2.png}
\end{minipage}%
}%
\subfigure{
\begin{minipage}[t]{0.33\linewidth}
\centering
\includegraphics[height=3cm]{figure/6.2.2.3.png}
\end{minipage}%
}%

\centering
\caption{在不同放置下静止状态的显示结果(红为X,绿为Y,蓝为Z)}\label{fig:6.2.2}
\end{figure}

\par{3)其他状态}
\par{对于其他状态,例如加速或者减速上坡,下坡这种变速行驶,主要是和冲击,自由落体这样类似的情况,不在做具体分析。但是由于传感器相当于是在一个黑匣子中感受外界,所以它只能反馈加速度的变化,而无法判断具体的方向——例如:无法判断自由下落和以-g的加速度减速上升的情况。}
\par{而对于匀速运动状态则与静止状况类似,可以进行类比分析。}

