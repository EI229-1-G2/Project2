\section{光电传感器“追光”功能实现}
\subsection{对问题进行分析}
\subsubsection{光电传感器与外界光照的关系?}
\par{对于实验板,其光电传感器位于下图所示位置,这两个光电传感器在测试性能上都是类似的:对于他们来说,光源越近,实际上的光照强度是越大的,而在实验板的反馈上来说,光照强度越强,所对应的IR是要越大的,所以其实可以通过IR来反馈外界光照强度的变化。

\subsubsection{如何通过光电传感器做到外界光照控制舵机的转动?}
\par{其实和上述原理分析类似,可以通过两个光电传感器的IR的值的差异来判断光源所在的位置,从而得到对光源的判断。而且更具两者之间的差异,可以通过一定的编程的方法,从而也因此可以转换成对舵机转向的控制。}

\subsubsection{如何合理的利用16位光柱判断方向角的动态实现?}
\par{和倒车雷达类似,倒车距离越近,倒车声音应该是要越急促的。所以对于这个来说,当转向向左时,灯应该是亮左边,向右时,灯应该向右边,而转弯的幅度越大,灯光所亮的应该越偏向转向的那一方。这样就能比较不错的反映出方向角的动态实现。}

\subsubsection{如何对光线太强或者太弱设置反映?什么算太强,什么算太弱?}
\par{其实想要将其展现出来,有很好的办法就是实验板的OLED板,可以直接将IR的数值显示在OLED板上,并且通过大小,显示不同的状态。但难点主要集中在什么算强光,什么又算弱光,这个只能通过实验来获得数据。我们将灯光分别置于中等距离,近距离和超远距离(无外加灯),分别对应中灯光,强灯光和弱灯光,得到两个分界线,分别是两个光电传感器IR的和小于30和大于300。}


\subsection{程序开发(配置工具的使用、信号处理算法的实现)}
\subsubsection{光电传感器控制舵机转动的实现}
\par{对应代码如下:}

\begin{lstlisting}[language = C]
IR1 = 0.9*IR1+0.1*AnalogIn.IRV1;
IR2 = 0.9*IR2+0.1*AnalogIn.IRV2;

IRread = (IR1-IR2)/(IR1+IR2);

tempInt = -IRread*800+1500;
\end{lstlisting}

\par{其实就是通过IRread这个式子从而得到光照所对应方位,并且将其值限定在了一个比值上,这样不但有利于判断对应的转动幅度,还可以对不同强度的光都能判断它的方位,因此可以有效地做到通过光源控制方向角,进而推动舵机转动。}

\subsubsection{16位光柱的显示}
\par{16位光柱显示代码如下:}

\begin{lstlisting}[language = C]
uint16_t a,b;
a=(tempInt/100)-7;
b=pow(2,a);
testCode=b;
BOARD_I2C_GPIO(testCode);
\end{lstlisting}

\par{和上述问题分析类似,是将tempInt转换为一个自变量,使得每当a变化时,可以通过一个对应的指数变化,使得b变化。再将b的数值显示在16位光柱上,从而使得tempInt每增加100,a就加1,16位光柱便向左偏移一格,从而实现方向角动态显示在16位光柱上。}

\subsubsection{光强的动态显示}
\par{光强的动态显示代码如下:}

\begin{lstlisting}[language = C]
if((IR1+IR2)>300)
{
    sprintf (OLEDLine4, "strong_Lt:%04hd",   AnalogIn.IRV1+ AnalogIn.IRV2);
}
else if((IR1+IR2)<30)
{
    sprintf (OLEDLine4, "feeble_Lt:%04hd",  AnalogIn.IRV1+ AnalogIn.IRV2);
}
else
{
    sprintf (OLEDLine4, "normal_Lt %04hd",  AnalogIn.IRV1+ AnalogIn.IRV2);
}

OLED_P8x16Str(0,6,(uint8_t *)OLEDLine4);
\end{lstlisting}

\par{显示实现原理比较简单,在此不做过多的解释}

\subsection{功能实现结果测试及分析(基于测试的设计方法,功能可靠性的分析及实现)}
\subsubsection{光照控制舵机转动和16位LED显示}
\par{可以看到,手机光源靠左时,舵机是朝左运动,并且LED光柱朝左移动,向右也是类似,可以看到实现效果还算不错,只是会有些抖动,这是因为手机是发散光源,在放置过程中可能会不太稳定。}

\begin{figure}[H]
\centering
\includegraphics[width=7cm]{figure/3.1.1.jpg}
\caption{光源在中间时舵机和灯都指在中间位置} \label{fig:3.1.1}
\end{figure}
\begin{figure}[H]
\centering
\includegraphics[width=7cm]{figure/3.1.2.jpg}
\caption{光源在左边时舵机和灯都指在左边位置} \label{fig:3.1.2}
\end{figure}
\begin{figure}[H]
\centering
\includegraphics[width=7cm]{figure/3.1.3.jpg}
\caption{光源在右边时舵机和灯都指在右边位置} \label{fig:3.1.3}
\end{figure}

\subsubsection{光照强度显示}
\par{可以看到,在OLED显示屏上面,会针对不同的光照强度,显示不同的状态,分别是“feeble\_Lt(弱光)”、“normal\_Lt(常光)”、“strong\_Lt(强光)”,并且在右侧会显示IR1+IR2的数值以便进一步的分析和完善,其实目前的分界还是比较粗略的,需要更多次的对比对照实验,应该可以进一步优化强度的显示。}

\begin{figure}[H]
\centering
\includegraphics[width=7cm]{figure/3.2.1.jpg}
\caption{“feeble\_Lt(弱光)”} \label{fig:3.2.1}
\end{figure}
\begin{figure}[H]
\centering
\includegraphics[width=7cm]{figure/3.2.2.jpg}
\caption{“normal\_Lt(常光)”} \label{fig:3.2.2}
\end{figure}
\begin{figure}[H]
\centering
\includegraphics[width=7cm]{figure/3.2.3.jpg}
\caption{“strong\_Lt(强光)”} \label{fig:3.2.3}
\end{figure}

