\section{分工与总结}
\par{通过本次实验,我们学习了旋转编码开关和I2C总线的工作原理,并实践应用,完成了简单的人机交互在嵌入式系统上的实现。}
\par{在任务一中,我们组首先结合上次课程中按键开关的知识,结合板载时钟,通过简单的状态机完成了实验一。阅读原框架代码锻炼了我们阅读和理解代码的能力。}
\par{在任务二中,学习旋转编码开关的工作原理和中断触发机制后,完成了代码的改动,并通过试验优化了OLED图形化显示的效果。但随即发现了抖动的问题,结合老师课程中讲解的几种方法进行实验,最后很好的完成防抖任务。}
\par{实验中,本组同学分工明晰,互帮互助,本次Lab1分工如下:}
\begin{table}[h]
	\centering
	\begin{tabular}{|l|c|}
		郝常升&代码编写,Latex汇总排版\\
		胡小雨&代码编写,“程序开发“、”功能实现展示”报告撰写\\
		张洋榕&代码编写,“问题分析“、”配置工具的使用”报告撰写\\
		高琛泰&烧录及代码调试,“问题整理”报告撰写\\
		李政&代码优化及调试,“传感器芯片工作原理”报告撰写\\
	\end{tabular}
	\caption{Group2分工}
	\label{tab:Margin_settings}
\end{table} 