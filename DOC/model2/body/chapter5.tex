\section{分工与总结}
\par{通过本次实验,我们学习了光电传感器和霍尔传感器的工作原理,并实践应用,利用光电传感器很好的实现了“追光”功能,并利用霍尔传感器很好的实现了“巡线”功能。}
\par{在任务中,我们组善用FreeMASTER先进行不同情况下(不同光强,距离,磁条位置,光照方向)的传感器信号特性分析,并用数学物理原理整合传感器反馈值,实现对不同状态的分别,将计算值映射到舵机转动角度。理论分析结束后,我们不断调试调整转动参数,使舵机转向和光照/磁条方向,最后做出了不错的结果。}
\par{同时,为了方便调试并使偏差值更加直观,我们设计了算法将偏差值以指数呈现在16位光条上,呈现效果即光条发亮位数与偏差方向呈较好的线性关系}
\par{实验中,本组同学分工明晰,互帮互助,本次Lab2分工如下:}
\begin{table}[h]
	\centering
	\begin{tabular}{|l|c|}
		郝常升&初版代码编写,Latex汇总排版\\
		胡小雨&霍尔代码编写,”霍尔传感器‘巡线’功能实现”报告撰写\\
		张洋榕&光电代码编写,“光电传感器‘追光’功能实现”报告撰写\\
		高琛泰&代码调试及数据采集,“霍尔传感器概述及信号特点”报告撰写\\
		李政&代码调试及数据采集,“光敏三极管概述及信号特点”报告撰写\\
	\end{tabular}
	\caption{Group2分工}
	\label{tab:Group2FG}
\end{table} 