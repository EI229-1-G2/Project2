\section{程序开发及配置工具使用}

\begin{figure}[H]
\centering
\includegraphics[width=16cm]{figure/4.1.1.png}
\label{fig:4.1.1}
\end{figure}

\subsection{程序实现}
\par{首先在程序中已经给出了CDK66\_Analog\_Input(\&AnalogIn)函数,即不断更新MMA7260三轴加速度传感器信号的输入,这样就可以保证我们得到的值为即时的。}
\par{之后我们再用AnalogIn类中的定义,对x,y,z方向的加速度进行全局变量定义,即}

\begin{lstlisting}[language = C++]
MemsX = AnalogIn.x;
MemsY = AnalogIn.y;
MemsZ = AnalogIn.z;
\end{lstlisting}

\par{这样就能把三轴方向上的加速度提取出来,通过OLED屏幕上的显示我们可知,当OLED屏幕朝上水平放置时,MemsX≈2000,MemsY≈1960,MemsZ≈3100(当OLED屏幕朝前或朝左时,MemsX或MemsY分别为3100左右,另外两个均为2000左右。故可知当MemsZ=3100时可说明板子水平放置。这一判断方法是由其重力加速度决定的:MemsX、Y、Z原本数值应均为2000左右,但水平放置时Z轴方向存在重力加速度,故MemsZ数值为3100,说明重力加速度带来的数值影响约1000左右),我们在代码中将加速度调零,即
}

\begin{lstlisting}[language = C++]
int absX = MemsX-2000,absY = MemsY-1960,absZ = MemsZ-2000;
\end{lstlisting}


\par{在这里我们可以看见将三轴加速度调整至0左右后,我们便能进行总加速度的公式求解}

\begin{lstlisting}[language = C++]
sqrt(absX*absX+absY*absY+absZ*absZ);
\end{lstlisting}

\par{由于板子静止放置时受到的板与加速度传感器的相对(重力)加速度带来的数值影响为1000左右;当实验板水平的自由落体时,板与加速度传感器的相对加速度为0,即原在Z轴方向存在的重力加速度也将消失,故MemsZ趋向于0(当实验板沿其它方向自由落体时也是同样的道理),则总加速度会骤降,我们可以利用骤降后的数值来判断实验板是在水平下降:}

\begin{lstlisting}[language = C++]
sqrt(absX*absX+absY*absY+absZ*absZ)<500;
\end{lstlisting}

\par{这里当总加速度小于500时即说明实验板是在做“跌落”。(这里采用的是当总加速度小于500时判断其为跌落,是因为我们在进行加速度调零时并不是十分准确的,故总加速度在自由落体时的数值肯定大于0,但数值也较小,故可用小于500来判断)。}
\par{同理我们可以分析,当实验板发生“冲击”时,其总加速度应该很大,在结合FreeMaster判断数值后,我们小组认为总加速度在大于1800时可以满足我们的要求,既不会特别敏感,也不会识别不出来是发生了“冲击”,故判断条件为}

\begin{lstlisting}[language = C++]
sqrt(absX*absX+absY*absY+absZ*absZ)>1800;
\end{lstlisting}

\par{而在实验要求中,我们需要通过蜂鸣器来判断是否发生“跌落”和“冲击”,即我们可以在判断语句后加上BEEP\_ON()来进行提示,同时当其“跌落”时将在数码管上显示1,当其发生“冲击”时将在数码管上显示2,这一功能通过调用ShowNumDEC()即可实现。综上,判断以及提示过程代码为}

\begin{lstlisting}[language = C++]
if (sqrt(absX*absX+absY*absY+absZ*absZ)<500){
    BEEP_ON();
    ShowNumDEC(1);
}
else if (sqrt(absX*absX+absY*absY+absZ*absZ)>1800){
    BEEP_ON();
    ShowNumDEC(2);
}
else {
	BEEP_OFF();
}

\end{lstlisting}

\par{综上,本次实验的代码实现部分为:}

\begin{lstlisting}[language = C++]
CDK66_Analog_Input(&AnalogIn);		// Update all analog inputs
MemsX = AnalogIn.x;
MemsY = AnalogIn.y;
MemsZ = AnalogIn.z;
int absX = MemsX-2000,absY = MemsY-1960,absZ = MemsZ-2000;
if (sqrt(absX*absX+absY*absY+absZ*absZ)<500){
    BEEP_ON();
    ShowNumDEC(1);
}
else if (sqrt(absX*absX+absY*absY+absZ*absZ)>1800){
    BEEP_ON();
    ShowNumDEC(2);
}
else {
	BEEP_OFF();
}

\end{lstlisting}


\subsection{配置工具的使用}
\par{在本次的Lab实验当中,除了照常的利用了MCUXpresso进行对应的程序编程之外,还在非常大的程度上利用了FreeMASTER进行一个数据的可视化的处理。在这次的CDK66\_KC4D工程文件里面,我们主要是对其相对三方向的加速度进行分析,而通过FreeMASTER,XYZ就可以很好的表现出来了。下面是此次实验FreeMASTER的使用流程。}
\par{(注意:在导入FreeMASTER之前,必须要将MCU中的程序成功烧录到单片机当中,并且在点击“继续”按钮之后维持运行一段时间,之后停止程序。在上述操作进行完之后,再进行FreeMASTER的操作才能正常继续。}
\\
\par{1)单片机的连接}

\begin{figure}[htbp]
\centering

\subfigure{
\begin{minipage}[t]{0.4\linewidth}
\centering
\includegraphics[height=4cm]{figure/3.1.1.1.png}
\end{minipage}%
}%
\subfigure{
\begin{minipage}[t]{0.4\linewidth}
\centering
\includegraphics[height=4cm]{figure/3.1.1.2.png}
\end{minipage}%
}%

\centering
\caption{单片机接口连接窗口}\label{fig:3.1.1}
\end{figure}

\par{2)导入所需变量}

\begin{figure}[H]
\centering
\includegraphics[width=5cm]{figure/3.1.2.png}
\caption{变量导入窗口} \label{fig:3.1.2}
\end{figure}

\par{3)创建示波器}

\begin{figure}[H]
\centering
\includegraphics[width=8cm]{figure/3.1.3.png}
\caption{示波器窗口} \label{fig:3.1.3}
\end{figure}

\par{4)生成动态图像}

\begin{figure}[H]
\centering
\includegraphics[width=12cm]{figure/3.1.4.png}
\caption{最终生成的示波器图像} \label{fig:3.1.4}
\end{figure}
