\section{霍尔传感器“巡线”功能实现}

\subsection{测试条件}
\par{首先我们根据实验器材安装可以判断出测试条件应为,将磁条平置不动,将实验板水平放置于磁条正上方一定距离,同时为了减小周围环境的干扰,应该远离电脑、手机等电磁设备。将实验板沿x轴左右平移,以此来实现霍尔电压值的改变,从而控制舵机转向。具体测试条件场景如图:}

\begin{figure}[H]
\centering
\includegraphics[width=8cm]{figure/HALL测试条件.jpg}
\caption{测试条件场景图} \label{fig:4.1}
\end{figure}

当磁条位于实验板下方中央时,舵机指向正前方,16位光柱显示在中央。

\subsection{代码实现及思路}
\par{我们将在case 6U部分(即四位拨码开关拨到0110)实现霍尔传感器控制舵机的功能。首先用\textbf{delay1ms(10)}来设置延时,以保证采样率的合理,再利用已给的\textbf{CDK66\_Analog\_Input(\&AnalogIn)}来读取霍尔传感器所测得的霍尔电压值所对应的信号值,即\textbf{AnalogIn.HALL1}以及\textbf{AnalogIn.HALL2}。}

\par{在一开始我们简单的设置了两个变量\emph{\textbf{H1=AnalogIn.HALL1,H2=AnalogIn.\-HALL2}},但在之后的实际测试中发现这样会产生很严重的抖动,即舵机转向一直不稳定,故经过尝试将其更改为\emph{\textbf{H1=0.9*H1+0.1*AnalogIn.HALL1}}和\emph{\textbf{H2=0.9*H2\-+0.1*AnalogIn.HALL2}},H1和H2的初值均为0,这样的设定让H1和H2参与进去,即通过闭环控制来改变我们所需要采集和利用的变量值,不像最开始的开环定义十分不精确,从而减小\textbf{AnalogIn.HALL1}和\textbf{AnalogIn.HALL2}对于我们需要采集的数据的抖动影响(直接影响),即在可靠的情况下使我们利用的数据更加稳定。经测试这样的变量设定确实可增加一个滤波功能,从而到达减小抖动的效果。}

\par{而为了实现根据H1和H2之间的关系式来确定舵机转动方向以及大小,我们首先在default模式中观察测试条件下的霍尔电压值对应的输出量,发现当实验板位于磁条的左右时其值会有明显不同:当磁条位于实验板右侧时,Mg1比Mg2大500\textasciitilde600左右;当磁条位于实验板左侧时,Mg2比Mg1大300\textasciitilde400左右。}

\par{基于测试对磁场信号太强(干扰)或太弱(丢线)的Mg1和Mg2,我们根据这些观察得来的经验数据,尝试了不同的\textbf{tempInt}函数关系式,比如\emph{\textbf{tempInt=H2-H1}}(此时会出现负值从而溢出的情况),\emph{\textbf{tempInt=H2-H1+1500}}(此时舵机转向大小不合适)等等函数,最终得出\emph{\textbf{tempInt = 3000*(H2-H1)/(H1+H2)+1500}}的函数式,可以较好的反映实验板位于磁条的左右位置从而合理调整舵机的角度。其中关于舵机转向以及角度的调整更新使用的是已有函数\textbf{Update\_ServoUS(kFTM\_Chnl\_0, tempInt)}和\textbf{Update\_ServoUS(kFTM\_Chnl\_1, 3000-tempInt)}。}

\par{除此之外根据实验要求,需要将识别的“位置偏差”动态显示于16位光柱,此时在case 6U中调用\textbf{BOARD\_I2C\_GPIO()}函数即可。又根据老师的要求,16位光柱作为显示舵机方向的直观现象:当舵机逆时针转时,16位光柱左边某一位变亮,当舵机顺时针转动时,光柱右边某一位变亮,且变亮的位置与舵机转动角度大小有关。为了实现这一功能,我们首先定义两个16进制数a和b,通过运算及分析得到公式\textbf{\emph{a=(tempInt/100)-7}}以及\emph{\textbf{b=pow(2,a)}},其中pow()是取指数的函数。再将b的值赋给testCode,进而显示在16位光柱上,即可很好的实现该功能。}

\par{此外关于霍尔磁场强度的直观标示,由于并没有十分强的磁场供我们使用,故在代码中仅根据经验设定一个上限,当超过此上限时,则在OLED屏幕上显示\textbf{strong\_MT}以及相应霍尔传感器对应值,同时在正常以及弱磁场情况下也进行的设定,再通过调用\textbf{OLED\_P8x16Str(0,6,(uint8\_t *)OLEDLine4)}函数进行OLED显示。}

\par{综上,整个霍尔传感器部分代码如下:}

\begin{lstlisting}[language = C]
case 6U:
    delay1ms(10);
    CDK66_Analog_Input(&AnalogIn);
    H1 = 0.9*H1+0.1*AnalogIn.HALL1;
    H2 = 0.9*H2+0.1*AnalogIn.HALL2;

    tempInt = 3000*(H2-H1)/(H1+H2)+1500;

    uint16_t a,b;
    a=(tempInt/100)-7;
    b=pow(2,a);
    testCode=b;
    BOARD_I2C_GPIO(testCode);

    if((H1+H2)>5000)
    {
        sprintf (OLEDLine4, "strong_MT:%04hd",   AnalogIn.HALL1+ AnalogIn.HALL2);
    }
    else if((H1+H2)<30)
    {
        sprintf (OLEDLine4, "feeble_MT:%04hd",  AnalogIn.HALL1+ AnalogIn.HALL2);
    }
    else
    {
        sprintf (OLEDLine4, "normal_MT %04hd",  AnalogIn.HALL1+ AnalogIn.HALL2);
    }
    OLED_P8x16Str(0,6,(uint8_t *)OLEDLine4);

    Update_ServoUS(kFTM_Chnl_0, tempInt);
    Update_ServoUS(kFTM_Chnl_1, 3000-tempInt);

    break;
\end{lstlisting}

\subsection{功能实现结果测试及分析}
\par{a.当磁条在实验板左侧时,舵机逆时针转动一定角度,16位光柱亮的位置在左侧。}

\par{b.当磁条在实验板右侧时,舵机顺时针转动一定角度,16位光柱亮的位置在右侧。}

\begin{figure}[htbp]
\centering

\subfigure{
\begin{minipage}[t]{0.5\linewidth}
\centering
\includegraphics[height=6cm]{figure/HALL磁条在实验板左侧.jpg}
\caption{HALL磁条在实验板左侧}\end{minipage}%
}%
\subfigure{
\begin{minipage}[t]{0.5\linewidth}
\centering
\includegraphics[height=6cm]{figure/HALL磁条在实验板右边.jpg}
\caption{HALL磁条在实验板右侧}\end{minipage}%
}%

\centering
\label{fig:4.2}
\end{figure}

\par{c.根据测试可以看到,舵机的抖动很小,随着实验板位置的缓慢连续变化可以几乎同时的产生相应转动,且方向也与磁条所在位置一直,从而达到“巡线”,很好的实现了要求。}

\par{d.可能存在的问题:}
\par{在实验中选取了周围环境中干扰较小的场景,且实验板与磁条的垂直距离较为固定,故在后续实验中将实验板装载在小车上后还需要更进一步的调整参数;}
\par{此外,关于巡线目的,实际巡线中舵机的转动角应更加精确。此次实验,我们仅作为一次尝试而设定了观察上较为合理的参数,后续实验需与巡线结合进行调参。}

