\section{霍尔传感器概述及信号特点}

\subsection{霍尔传感器}
\par{利用霍尔效应制成的各种传感器可用于磁场强度、运动及各种特殊位置的检测。作为一种非接触式传感器,霍尔效应传感器的输出电压与磁场强度成比例,而与磁场变化率无关。作为低速传感器,相较于电感式传感器与接触式传感器其更可靠也更便携。}

\subsubsection{AD22151传感器信号特性}
\par{由ADI推出的AD22151是一种线性输出磁场传感器,其输出电压与施加在垂直器件封装顶面的磁场强度成正比。它将大量霍尔元件阵列集成技术与内部温度补偿及信号调节电路结合起来实现单片集成。这种传感器根据对具体信号的应用要求设置增益而且增益调整范围很宽。输出电压可以调整,既可检测双极性磁场,也可检测单极性磁场。每种工作方式的信号输出幅度都与电源电压成比例。在内部温度补偿电路的控制下,不但利用内部补偿电阻进行二级补偿,而且还可利用外界电阻进行一级补偿。由下图可知,在经过信号调理与温度补偿后,该元件具有较好的线性性。}

\subsection{可靠识别中心位置偏差}

\begin{figure}[htbp]
\centering

\subfigure{
\begin{minipage}[t]{0.5\linewidth}
\centering
\includegraphics[height=4.5cm]{figure/Mg1.png}
\caption{霍尔传感器1信号特性}\end{minipage}%
}%
\subfigure{
\begin{minipage}[t]{0.5\linewidth}
\centering
\includegraphics[height=4.5cm]{figure/Mg2.png}
\caption{霍尔传感器2信号特性}\end{minipage}%
}%

\centering
\label{fig:2.2}
\end{figure}

\par{由上两图可知,在误差允许的情况下,距实验板中心越近,霍尔传感器可检测到的磁场强度就越弱,这是由于实验板中,霍尔传感器位于实验板两侧(此实验中,以横坐标为0处为Hall\_Sensor\_1所在位置,实验板长约为25cm,则横坐标为25处为Hall\_Sensor\_2所在位置)。可看出,该种检测思路不仅能可以检测出距板中心的偏差,也可以检测出偏差方向,具有较好的检测效果。}

\subsection{避免磁场强弱变化对识别结果的影响}
\par{由上两图可知,在误差允许的情况下,无论对Hall\_Sensor\_1还是Hall\_Sensor\_2,实验板距磁条垂直距离对磁场强度均有比较大的影响,尤其是当实验板距磁条垂直距离较近时,其影响较大。利用这种影响,在实验时,我们采取板贴磁条移动(即保持垂直距离为0)进行实验,可避免磁场在垂直板方向的变化对检测效果的影响。
此外,在实验时,我们也采取了归一化策略即:$$ \delta H = \frac{H1-H2}{H1+H2} $$这样可避免磁场强度绝对值的大小对跟随效果的影响。}
