\pagenumbering{arabic}
\section{光敏三极管概述及信号特点}

\subsection{光敏三极管原理}
\par{光敏三极管,也称光电三极管或光电晶体管,它是作为光传感器的敏感部分,已在光的检测、信息的接受、传输、隔离等方面获得广泛的应用,成为各行各业自动控制必不可少的器件。其基本原理是光照到P-N结上时,吸收光能并转变为电能。当光敏三极管加上反向电压时,管子中的反向电流随着光照强度的改变而改变,光照强度越大,反向电流越大,大多数都工作在这种状态。}

\subsection{信号状态分析}

\subsubsection{光线强弱分析}
\par{由光敏三极管LTR-546AD特性可知,当光线太弱是,反向电流检测效果不好。光敏三极管的感知特性也光照强度和光的波长有关,现在讨论对不同光照条件下光敏三极管的IR值,以判断光线强弱。}
\par{设定光敏三极管1的IR值为IR1,光敏三极管2的IR值为IR2,选取(IR1+IR2)/2为光照强度,测试结果如下:}

\begin{table}[h]
	\centering
	\begin{tabular}{|l|c|}
        照明情况&(IR1+IR2)/2\\
        一般室内灯光照明&20-72\\
        晴天室外&2800-3700\\
        手机手电筒照射(在一般室内灯光照明下)&100-900\\
	\end{tabular}
	\caption{不同照明情况}
	\label{tab:1.1}
\end{table}


\begin{table}[h]
	\centering
	\begin{tabular}{|c|c|}
        光源与设备距离(cm)&(IR1+IR2)/2\\
        11&32\\
        9.4&40\\
        7.5&55\\
        5.4&85\\
        3.4&160\\
        1.3&310\\
        0.4&1250\\
	\end{tabular}
	\caption{不同距离\emph{IR1+IR2/2}的值}
	\label{tab:1.2}
\end{table}

\subsubsection{可靠识别光源相对方位角的方法}
\par{经过两个光敏三极管的配合,通过识别两个光敏三极管传回的IR值能很好地判断光源相对位置。通过两个光电传感器的IR值的差异来判断光源所在的位置,从而得到对光源的判断。由此得到光源相对实验板的方位角。如下图所示:}

\begin{figure}[H]
\centering
\includegraphics[width=12cm]{figure/1.2.1.png}
\caption{光场示意图} \label{fig:1.2.1}
\end{figure}

\par{X为光源到光敏三极管1的距离,y为光源到光敏三极管2的距离,通过(x-y)的值以及一系列运算即可算出光源的位置,并得出光源相对实验板的方位角。}

\subsubsection{如何选择合适的光源}
\par{由光敏三极管LTR-546AD特性可知,光源波长在900nm左右三极管有最高的敏感度。选取光的波长在900nm,光照强度足够大的光源。}

\begin{figure}[H]
\centering
\includegraphics[width=12cm]{figure/1.1.1.png}
\caption{光敏三极管LTR-546AD特性} \label{fig:1.1.1}
\end{figure}

