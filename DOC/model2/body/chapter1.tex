\pagenumbering{arabic}
\section{问题整理及应用实例}

\subsection{实验内容要求}
\par{1.配置接口,采集MMA7260三轴加速度传感器信号并显示}
\par{2.分析三轴加速度信号在不同运动状态下的特征}
\par{3.设计数据处理方法,准确识别“跌落”状态并通过蜂鸣器提示,并区分“冲击”等不同场景的区别处理
}

\subsection{自由落体检测}
\par{自由落体指常规物体只在重力作用下,初速度为0的运动。在本实验的实验情境下,其主要是指只在重力作用(忽略空气阻力等作用)下,运动对象(即开发板)开始速度向下等运动。在三轴加速度传感器下,检测竖直方向的加速度即可。但还会出现如“冲击”等相似运动,需要区分}

\subsection{跌落检测前景}
\par{1.在电子器械尤其是精密电子器械的日常使用与运输过程中,可能会出现从高处跌落情况。人们日常生活中不可或缺的智能手机,目前也已加入三轴加速度传感器模块,用来检测可能的跌落,并向使用者报告,提供相应的售后点使消费者可及时修复手机。}
\par{2.老年人易出现跌倒、摔落等问题,且跌倒后难以与家人、医院等取得联系。利用可穿戴式的加速度传感器,并利用互联网与外界进行交流,可以及时对老年人的跌倒引起的受伤进行及时救治。}

\subsection{基于多传感器融合的跌倒检测实例}
\par{目前,许多研究人员采用多传感器融合技术来检测人体的运动情况,其主要通过使用如三轴加速度传感器、三轴振动式角速度传感器(陀螺仪)等,获取多项运动数据信息。而对于跌倒的检测,则主要分为两个方向:基于阈值分析等跌倒检测与基于机器学习分类的跌倒检测研究。}

\par{对于上述的传感器,其经过简单的可穿戴改造后,即可用于测量穿戴者运动时在X轴、Y轴和Z轴方向上的加速度与角速度。利用加速度度传感器,可计算载体的速度、位移信息,联合角速度传感器则可以有效记录人体在行动过程中实时姿态的变化。利用LIS3DSH、L3GD20、BMA250等体积小、可靠性高、灵敏度高、易于集成的传感器,可以在低成本的情况下得到有关人体姿态的各项信息。}

\par{基于阈值分析的跌倒检测主要处理流程包括设置一个或多个阈值来判断是否出现跌倒行为,传感器接收的数据信号若突破了预设的阈值,则判断为跌倒。其运算量较小,响应速度较快,在显示老人跌倒的紧急情况下更有实用价值。}

\begin{figure}[htbp]
\centering

\subfigure{
\begin{minipage}[t]{0.5\linewidth}
\centering
\includegraphics[height=4cm]{figure/watch.jpg}
\end{minipage}%
}%
\subfigure{
\begin{minipage}[t]{0.5\linewidth}
\centering
\includegraphics[height=4cm]{figure/1.1.2.jpg}
\end{minipage}%
}%

\centering
\caption{跌落检测应用示意图}\label{fig:1.1.1}
\end{figure}

\par{对于基于机器学习分类的跌倒检测方法,在获得原始传感器数据后,通常需要经过滤波去噪、特征向量提取和机器学习分类这三个数据处理流程,才能获得关于被测人员是否跌倒的判断。其发展较为迅速,不同的研究人员基于机器学习算法设计了不同的研究,普遍取得了较高的准确率。}
\par{两类跌倒检测算法将会结合更多的实际应用背景。首先目前跌倒检测研究的数据多源于实验室,是传感器正确佩戴、测试人员人为跌倒的结果,数据信息采集过程极为理想化,脱离实验室环境后难以落地部署。其次当下研究人员鲜少考量到人体日常活动与跌倒之间较为模糊的状态,例如快速蹲下和快速躺下,这类动作既可能来源于人体正常日常活动,也可能来自于跌倒过程,但大部分研究都人为得将其直接归类为“跌倒”或“未跌倒”。人体的运动行为是较为复杂和随机的,只针对部分特征向量来实验,会导致出现大量漏报、误报的情况。在现实生活中,漏报会给老人的生命安全带来严重影响,而误报则会产生不必要的人力、时间资源的浪费。所以未来研究人员应该注重算法在现实老年人跌倒场景下的泛化能力,可以在不同危急情况下得到准确的结果。}


