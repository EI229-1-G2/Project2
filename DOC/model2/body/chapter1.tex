\pagenumbering{arabic}
\section{问题整理}

\subsection{实验内容要求}
\par{1.配置接口,采集MMA7260三轴加速度传感器信号并显示}
\par{2.分析三轴加速度信号在不同运动状态下的特征}
\par{3.设计数据处理方法,准确识别“跌落”状态并通过蜂鸣器提示,并区分“冲击”等不同场景的区别处理
}

\subsection{自由落体检测}
\par{自由落体指常规物体只在重力作用下,初速度为0的运动。在本实验的实验情境下,其主要是指只在重力作用(忽略空气阻力等作用)下,运动对象(即开发板)开始速度向下等运动。在三轴加速度传感器下,检测竖直方向的加速度即可。但还会出现如“冲击”等相似运动,需要区分}

\subsection{跌落检测前景}
\par{1.在电子器械尤其是精密电子器械的日常使用与运输过程中,可能会出现从高处跌落情况。人们日常生活中不可或缺的智能手机,目前也已加入三轴加速度传感器模块,用来检测可能的跌落,并向使用者报告,提供相应的售后点使消费者可及时修复手机。}
\par{2.老年人易出现跌倒、摔落等问题,且跌倒后难以与家人、医院等取得联系。利用可穿戴式的加速度传感器,并利用互联网与外界进行交流,可以及时对老年人的跌倒引起的受伤进行及时救治。}
