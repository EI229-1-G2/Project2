\section{功能实现结果测试及分析}
\subsection{功能实现情况}
\subsubsection{自由落体测试}
\par{1)首先当实验板自由落体时,我们可以听到明显的蜂鸣声,且一直伴随着实验板的“跌落”而提示,且数码管上显示1,表明其是在自由落体而不是受到“冲击”。(下面图片为“跌落”实现结果观察过程的简单展示)}



\subsubsection{冲击测试}
\par{1)其次为当实验板受到“冲击”时,我们也能听到明显的蜂鸣声,且数码管显示2,此时蜂鸣声明显较短,说明冲击仅在较短时间内发生。(下面图片为“冲击”实现结果观察过程的简单展示)}



\subsection{可能存在的问题及解决方法}
\par{在代码层面我们可以看出的明显的问题是:}
\par{(1)关于三轴加速度的调零问题。即在将MemsX,MemsY,MemsZ减去一个数值再分别赋给absX,absY和absZ时,减去的数值虽然不同,但也是我们根据人为经验观察所得出的,故肯定存在较为明显的误差,这也是为什么我们在判断“跌落”时将判断条件定为(我们设定的)总加速度小于500。}
\par{尝试解决的方法:由于MemsX,MemsY,MemsZ的数值一直在变化,但我们可以尝试将其数值导入外部统计软件中再进行计算均值,根据均值再去进行调零,这样能在一定范围内提高精度。但是毫无疑问,由于这三个值在不断变化,故找不到一个很好的方法进行准确调零。}
\par{(2)在判断条件中,“跌落”是总加速度小于500,“冲击”是总加速度大于1800,但500和1800也都是我们人为经验判断得出的,故在灵敏度上肯定存在一定误差。}
\par{尝试解决的方法:利用FreeMaster等可视化数值软件得出在“跌落”和“冲击”时MemsX,MemsY,MemsZ的准确值,再通过公式计算得出判断条件的数值大小,这样能在一定范围内提高精确度。}

